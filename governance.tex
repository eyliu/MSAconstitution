\article{Student Governance}
\section{Authority}
    All authority under this Constitution is derived from the students of the University of Michigan.

\section{Governments}
    All student governance powers granted herein shall vest in a Central Student Government. The Central Student Government shall be democratically constituted, and shall consist of central legislative, executive, and judicial powers. All student governance powers not granted herein shall devolve to the additional governments for each school or college, University-owned and operated housing (and for each individual house and building therein), fraternities, sororities, and cooperatives; and for such jurisdictions as the smallest government containing the jurisdiction shall certify. The creation, structure, functions, and operations of each government shall be determined by the government's student constituency.

    Each student government or organization, each housing unit, and each federation of student governments, organizations, or housing units shall have a constitution approved by the students within the jurisdiction of the government, organization, housing unit, or federation providing for the democratic selection of its leadership and representative and democratic policy making within the government, organization, housing unit, or federation. Each such democratically constituted government shall be the governmental representative, legislative, and coordinating organization of the students of that jurisdiction, shall, upon a majority vote of its student constituents levy dues, and provide for their collection equally from each of the student constituents of the government; shall appropriate its own dues money and such other income as it shall receive; shall make appointments of student representatives to all student seats on committees whose purview is coextensive with the jurisdiction containing that committee; and  shall conduct its elections so as to insure that its constituents are given ample opportunity to cast their ballots, that the election is free from fraud and that open campaigning can take place.

    All general sessions of student government bodies recognized under this Constitution shall be open to students at large.

\section{Elections.}
    The Central Student Government shall have the power to hold elections for its offices and for referenda, coordinate with other governments elections for the offices of those governments, and regulate campaign practices on campus. Elections for all Central Student Government offices shall be held each twice each year, once in November and once in March. Each term of office shall begin two weeks after the election in which the seat was filled. Elections shall be administered and certified by an University Elections Committee, which will serve at the direction of the Student General Council. The Student General Council shall have the authority to hire an elections administrator.

\subsection{Apportionment.}
For bodies represented by a proportional count of students, including the Student Assembly, the Student General Council shall present to the Assembly and the students the apportionment for each such body no less than one month before each scheduled election.

\subsection{Election of Assembly representatives.}  Assembly Representatives shall be elected in the March election for a one year term.

\subsubsection{Vacancies.}  If a representative's seat is vacated, then the respective school or college government may appoint a student currently enrolled in the respective school or college to hold that seat until the next Central Student Government election.  If there is not a school or college government for the respective seat, then the Dean of the respective school or college may appoint a student currently enrolled in the respective school or college to hold that seat until the next Central Student Government election.
\subsubsection{Method of voting.}  A student may only vote for candidates currently enrolled in his/her degree-granting unit.  A student enrolled in more than one degree-granting unit may only vote for candidates in one of those constituencies.  The method of voting for Assembly representatives shall be the Borda count.  Each voter may vote for $n$ candidates in his/her constituency, where $n$ is the number 
of seats open in the constituency. The voter shall rank the candidates from 
1 to $n$ on the basis of preference.  A $k$th place vote shall count for $n-k+1$ points, such that a first place vote shall count for $n$ points, where $n$ 
is the number of seats in the constituency, a second place vote shall count 
for $(n-1)$ points, and so forth, such that an $n$th place vote shall count for one point.  The $n$ candidates with the largest total vote counts shall be declared the winners.  In the event of a tie, the newly elected Assembly shall choose amongst the tied candidates.

\subsection{Election of the Student Body President.}
The Student Body President and Vice President shall be elected together in the March election for a one year term by a vote of the student body at large.  The method of voting for the Student Body President and Vice President shall be the method of plurality voting.  Each student will be allowed to vote for one slate, and the slate with the most votes shall be declared the winners.  In the event of a tie, the newly elected Assembly shall choose amongst the tied candidates.