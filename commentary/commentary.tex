\documentclass[12pt,letterpaper]{article}
\usepackage[margin=1in]{geometry}
\usepackage{mathpazo}
\usepackage{todonotes}
\usepackage{lineno}
\usepackage{xspace}
\begin{document}
\title{Commentary on the Constitution of the University of Michigan Student Government (UMSG)}
\author{Publius}
\date{April 1, 2010}
\maketitle


\section*{What should student government do?}
\begin{itemize}
	\item Self-governance of the student body of an educational institution.
	\begin{itemize}
		\item Make rules.
		\item ``Appoint'' officials.
		\item Regulate subsidiary student governments (only to ensure that these reflect egalitarian and democratic principles) and student organizations.
	\end{itemize}
	\item Represent the interests of the student body to other bodies (regents, administration, faculty, city/state/federal governments)
	\item Student projects and activities.
	\begin{itemize}
		\item Collect student fees
		\item Appropriate money
	\end{itemize}
\end{itemize}


\section*{Flaws in the previous constitution}

\begin{enumerate}
	\item Lack of separation of powers, specifically conflation of executive and legislative powers.
	\begin{itemize}
		\item Exhibit A: The Steering Committee.  Is its function legislative or executive?  Why does it exist?  Why does it have the powers that it's been granted?  Why does the Assembly routinely circumvent the normal legislative process.
	\end{itemize}
	\item The details of election procedures are not clear to voters, and may be unfair to independent candidates.
	\begin{itemize}
		\item Single-winner elections are conducted using plurality voting.  The only virtue of plurality voting is simplicity.  Plurality voting fails the criterion of independence of irrelevant alternatives (the ``Nader effect'') and tends to result in a two-party system (``Duverger's law'').
		\todo[inline]{Enumerate the other flaws of plurality voting.}
		\item Multiple-winner elections are conducted using the Borda count.  The Borda count in normally applied to single-winner elections, and even then, it is usually cited as the canonical example of a voting system that is extremely susceptible to tactical voting.
	\end{itemize}
\end{enumerate}


\section*{The current structure}

\subsection*{The Players}
	\begin{itemize}
		\item The Assembly (legislative)
		\item Executives
		\item (Legislative?) Committees \& (Executive?) Commissions
		\item Central Student Judiciary
	\end{itemize}

The chairs of the committees \& commissions form the Steering Committee, which meets weekly.  The Steering Committee sets the Assembly's agenda and can allocate a limited amount of money.  The current constitution doesn't actually specify to which branch the Steering Committee or the committees \& commissions belong.

\subsection*{Bizarre features of the current structure}
\begin{itemize}
	\item The President and Vice President are defined to be legislators.  The other executives may or may not be legislators.
	\item The Steering Committee is predominantly comprised of (executive?) commissions, but serves a predominantly legislative function.
\end{itemize}

\section*{Principles of representation}

\begin{enumerate}
	\item Every student should be guaranteed representation in student government.
	\item Representation in the Assembly should be as proportionate as possible.
	\item Representatives should be democratically-elected.
\end{enumerate}

\end{document}