\documentclass[12pt,letterpaper]{article}
\usepackage[margin=1in]{geometry}
\usepackage{mathpazo}
\usepackage{todonotes}
\usepackage{lineno}
\usepackage{xspace}
\begin{document}
\title{Commentary on the Constitution of the University of Michigan Student Government (UMSG)}
\author{Publius}
\date{April 1, 2010}
\maketitle


\section*{What should student government do?}
\begin{itemize}
	\item Self-governance of the student body of an educational institution.
	\begin{itemize}
		\item Make rules.
		\item ``Appoint'' officials.
		\item Regulate subsidiary student governments (only to ensure that these reflect egalitarian and democratic principles) and student organizations.
	\end{itemize}
	\item Represent the interests of the student body to other bodies (regents, administration, faculty, city/state/federal governments)
	\item Student projects and activities.
	\begin{itemize}
		\item Collect student fees
		\item Appropriate money
	\end{itemize}
\end{itemize}


\section*{Flaws in the previous constitution}

\begin{enumerate}
	\item Lack of separation of powers.
	\item The details of election procedures are not clear to voters, and may be unfair to independent candidates.
\end{enumerate}


\section*{Principles of representation}

\begin{enumerate}
	\item Every student should be guaranteed representation in student government.
	\item Representation in the Assembly should be as proportionate as possible.
	\item
\end{enumerate}

\end{document}