\article{The Central Legislature}

\section{The Legislature}
    The central student legislature shall consist of a Student Assembly and a Senate.


\section{The Assembly}
    The Assembly shall be composed of representatives elected from among the students every year according to their degree-granting unit. Each degree granting-unit shall have one representative for every 800 students enrolled in that unit, determined by the average of the most recent Winter Term and Fall Term enrollment of each unit. Such apportionment shall be tabulated once each year. Each unit shall have at least one representative. If, after the total enrollment of the unit is divided by 800, the remainder is greater than 400, the unit shall be granted an additional representative. Each representative shall have one vote in the Assembly. Vacancies.

    The Assembly shall have the power to elect its own Speaker, Vice Speaker, and other officers as it shall deem necessary from among its membership. The Speaker shall chair general sessions of the Assembly and serve on the Central Student Government's Executive Committee. Assembly officers shall be responsible for accepting items for the Assembly's agenda and docket, and officers may be recalled by a two-thirds majority vote of the Assembly.

     The Assembly shall have the power to levy dues and provide for their collection equitably among the students. The Assembly shall not raise the level of the fee above a maximum limit approved by a vote of the student body and the Regents. It shall appropriate all funds collected within the fee limit to student organizations and student body programs and events as it shall deem expedient. All such appropriations shall be subject to the approval of the Senate without their amendment. All funds collected in excess of the fee limit shall be placed in a University account created for the sole purpose of holding such funds, and shall only be appropriated upon a referendum empowering the Central Student Government to appropriate such funds.

    The Assembly shall have the power to pass resolutions and, with the agreement of a simple majority of the Senate, pass any amendments to the Compiled Code other document binding upon students or student organization. Also with the agreement of a simple majority of the Senate, the Assembly may place proposed amendments to this constitution before the students for ratification according to the process described in Article .

    The Assembly shall form several committees to execute its duties. It shall maintain standing committees on the budget, rules, and communication. It may create additional standing committees by with a simple majority vote of the Assembly and the approval of a simple majority of the Executive Committee. The Assembly may also create special committees to serve for a specified term by a simple majority of the Assembly. Any Assembly committee may be dissolved by a two-thirds majority vote of the Assembly.


\section{The Senate}
    The Senate shall be composed of senators chosen by the several college, school and organization governments every year. Two senators shall be elected by the membership of each such government from among their number or by such other method as they shall determine. Each senator shall have one vote in the Senate.

    The Central Student Government Vice President shall serve as president of the Senate, but shall have no vote, unless the Senate shall be equally divided. The Senate shall choose other officers as it shall deem necessary from among its members. Senate officers shall be responsible for accepting items for the Senate's agenda and docket, and officers may be recalled by a two-thirds majority vote of the Senate, excepting the Vice President.

    The Senate shall have the power to pass resolutions and, with the agreement of a simple majority of the Assembly, pass any amendments to the Compiled Code other document binding upon students or student organizations. Also with the agreement of a simple majority of the Assembly, the Senate may place proposed amendments to this constitution before the students for ratification according to the process described in Article.

    The Senate shall form several committees to execute its duties. It shall maintain standing committees on the community service, campus governance, and external relations. It may create additional standing committees by with a simple majority vote of the Senate and the approval of a simple majority of the Executive Committee. The Senate may also create special committees to serve for a specified term by a simple majority of the Senate. Any Senate committee may be dissolved by a two-thirds majority vote of the Senate.


\section{Proceedings and Membership}
    The Assembly and Senate may each determine rules of its own proceedings.

    The Assembly and Senate shall each keep a journal of its proceedings which shall be made available in a timely fashion to the students at large.

    No elected officer of the Central Student Government shall, within his or her elected term, be appointed to any other office in the Central Student Government.

    Any elected or appointed officer of the Central Student Government may be removed from office for delinquency, corruption, or other derelictions. Articles of impeachment must pass the Assembly by a simple majority, after which they shall be presented to the Senate for a hearing. When the Senate is convened to hear impeachment charges, the Chief Justice of the Central Student Judiciary shall preside. At the conclusion of the hearing, a two-thirds majority of the Senate shall be required for conviction. A convicted officer shall be immediately removed from office.


\section{Legislative Process and Veto}
    The Assembly or Senate may pass resolutions without the consent of the other house by a simple majority vote, but if such a resolution concerns an amendment to the Complied Code or would be otherwise binding upon students or student organizations, the resolution must meet the approval of a simple majority of both houses, then be submitted to the President. The President may, upon receipt of such resolution, veto it within one week and return it to the Central Student Legislature for reconsideration. After such reconsideration, the President's veto may be overridden by a two-thirds majority of the Assembly and a two-thirds majority of the Senate.

