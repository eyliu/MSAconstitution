\article{The Central Legislature}
\section{The Legislature}
    The central student legislature shall consist of a Student Assembly and a University Council.

\section{The Assembly}
    The Assembly shall be composed of representatives elected from among the students every year according to their degree-granting unit. Each degree granting-unit shall have one representative for every 800 students enrolled in that unit, determined by the average of the most recent Winter Term and Fall Term enrollment of each unit. Such apportionment shall be tabulated once each year. Each unit shall have at least one representative. If, after the total enrollment of the unit is divided by 800, the remainder is greater than 400, the unit shall be granted an additional representative. Each representative shall have one vote in the Assembly. The Assembly shall meet at least once each week during the academic year.

    Any student organization with at least 400 active members currently enrolled at the university may apply for an Ex-Officio seat on the Assembly. The Assembly shall create, publish, and maintain requirements for review of such applications.

    The Assembly shall have the power to elect its own Speaker, Vice Speaker, and other officers as it shall deem necessary from among its membership. The Speaker shall chair general sessions of the Assembly and serve on the Central Student Government's Executive Committee. Assembly officers shall be responsible for accepting items for the Assembly's agenda and docket, and officers may be recalled by a two-thirds majority vote of the Assembly.

     The Assembly shall have the power to levy dues and provide for their collection equitably among the students. The Assembly shall not raise the level of the fee above a maximum limit approved by a vote of the student body and the Regents. It shall appropriate all funds collected within the fee limit to student organizations and student body programs and events as it shall deem expedient. All such appropriations shall be subject to the approval of the Senate without their amendment. All funds collected in excess of the fee limit shall be placed in a University account created for the sole purpose of holding such funds, and shall only be appropriated upon a referendum empowering the Central Student Government to appropriate such funds.

    The Assembly shall maintain the Compiled Code in order to exercise the powers and carry out the functions described herein.

    The Assembly shall maintain standing committees on Campus Governance, External Relations, Rules and Elections, Budget Priorities, Community Service, and Communications. At its discretion, the Assembly may form select committees by a majority vote. Select committees shall serve for one year, but may be reconvened for subsequent terms by a majority vote of the Assembly. Committees shall be chaired by an elected representative of the Assembly.

    The Assembly shall have the power to pass resolutions and pass any amendments to the Compiled Code or other document binding upon students or student organization. With the agreement of a simple majority of the Senate, the Assembly may place proposed amendments to this constitution before the students for ratification according to the process described in Article V of this Constitution.

    The Assembly shall form several committees to execute its duties. It shall maintain standing committees on the budget, rules, resolutions and communication. It may create additional standing committees by with a simple majority vote of the Assembly and the approval of a simple majority of the Executive Committee. The Assembly may also create special committees to serve for a specified term by a simple majority of the Assembly. Any Assembly committee may be dissolved by a two-thirds majority vote of the Assembly.

\section{The University Council}
    The University Council shall be composed of representatives chosen by the several college, school and organization governments every year. One representative shall be elected by the membership of each such government from among their officers or by such other method as that government shall determine. Each representative shall have one vote in the Council. The Council shall meet at least twice each month during the academic year.

    The Central Student Government Vice President shall serve as president of the Council, but shall have no vote, unless the Council shall be equally divided. The Council shall choose other officers as it shall deem necessary from among its members. Council officers shall be responsible for accepting items for the Council's agenda and docket, and, excepting the Vice President, officers may be recalled by a two-thirds majority vote of the Council.
  
    With the agreement of a simple majority of the Assembly, the Council may place proposed amendments to this constitution before the students for ratification according to the process described in Article V.

    Any student organization with at least 400 active members currently enrolled at the university may apply for an Ex-Officio seat on the Council. The Council shall create, publish, and maintain requirements for review of such applications.

    The Council shall form such committees as it deems necessary to execute its duties.

    The Council shall oversee the University Elections Commission for administering and certifying elections to the Central Student Government. The University Elections Commission shall be chaired by the Student General Council.

    Each school, college, or organization government recognized by membership in the Council shall, from time to time, be able to recommend to the Council such measures as it deems necessary and expedient. After deliberation on such a measure, the Council may, by a simple majority, add the measure at issue to the agenda of the next Assembly meeting. No more than one such measure shall have originated from any single Council member per Council meeting.

\section{Proceedings and Membership}
    The Assembly and University Council shall each determine rules of its own proceedings.

    The Assembly and University Council shall each keep a journal of its proceedings which shall be made available in a timely fashion to the students at large.

    No elected officer of the Central Student Government shall, within his or her elected term, be appointed to any other office in the Central Student Government.

    Any elected or appointed officer of the Central Student Government may be removed from office for delinquency, corruption, or other derelictions. Articles of impeachment must pass the Assembly by a simple majority, after which they shall be presented to the University Council for a hearing. When the Council is convened to hear impeachment charges, the Chief Justice of the Central Student Judiciary shall preside. At the conclusion of the hearing, a two-thirds majority of the Council shall be required for conviction. A convicted officer shall be immediately removed from office.

    If at any time a seat on the Assembly or University Council shall become vacant, the legislature of the inferior constituency controlling the seat shall appoint a new representative in such manner as it shall deem appropriate. The appointed representative shall serve until the next scheduled election, at which time the voters of that constituency shall fill the vacancy by election.

\section{Legislative Process and Veto}
    If the Assembly passes a resolution concerning an amendment to the Complied Code or a resolution that would be otherwise binding upon students or student organizations, that resolution must then be submitted to the President. The President may, upon receipt of such resolution, veto it within one week and return it to the Assembly for reconsideration. After such reconsideration, the President's veto may be overridden by a  two-thirds majority of the Assembly.