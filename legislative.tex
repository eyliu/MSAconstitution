\article{The Central Legislature}
\section{The Legislature.}
    The central student legislature shall consist of a Student Assembly and a University Council.

\section{The Assembly.}
    The Assembly shall be composed of representatives elected from among the students every year according to their degree-granting unit, as defined by the University's Board of Regents. Each degree granting-unit shall have one representative for every 800 students enrolled in that unit, determined by the average of the most recent Winter Term and Fall Term enrollment of each unit. Such apportionment shall be tabulated once each year. Each unit shall have at least one representative. If, after the total enrollment of the unit is divided by 800, the remainder is greater than 400, the unit shall be granted an additional representative. Each representative shall have one vote in the Assembly.

    Assembly Representatives shall be elected in the March election for a one year term.  The method of voting for Assembly representatives shall be a Borda count. Each voter may vote for no more than $n$ candidates in his/her constituency, where $n$ is the number of seats open in the constituency. The voter shall rank the candidates from 1 to $n$ on the basis of preference.  A $k$th place vote shall count for $(n - k + 1)$ points, such that a first place vote shall count for $n$ points, a second place vote shall count for $(n - 1)$ points, et cetera, such that an $n$th place vote shall count for one point.  The $n$ candidates with the most total points shall be declared the winners.  In the event of a tie, the newly elected Assembly shall choose amongst the tied candidates.

    Any student organization with at least 400 active members currently enrolled at the university may apply for a non-voting ex-officio seat on the Assembly. The Assembly shall create, publish, and maintain requirements for review of such applications.

    The Assembly shall meet at least weekly during the academic year.

    The Assembly shall have the power to elect its own Speaker, Vice Speaker, and other officers as it shall deem necessary from among its membership. The Speaker shall chair general sessions of the Assembly and serve on the Central Student Government Executive Committee. Assembly officers shall be responsible for accepting items for the Assembly's agenda and docket, and officers may be recalled by a two-thirds majority vote of the Assembly.

     The Assembly shall have the power to levy dues and provide for their collection equitably among the students. The Assembly shall not raise the level of the fee above a maximum limit approved by a vote of the student body and the Regents. It shall appropriate all funds collected within the fee limit to student organizations and student body programs and events as it shall deem expedient. All funds collected in excess of the fee limit shall be placed in a University account created for the sole purpose of holding such funds, and shall only be appropriated upon a referendum empowering the Central Student Government to appropriate such funds. 

    The Assembly shall produce, publish, and maintain Operating Procedures, which shall describe the standing rules, procedures, and internal structures of the Assembly. The Operating Procedures shall provide for the manner of officer election, procedures for the formation and operation of committees, and any other rules of Assembly proceedings necessary for the execution of Assembly duties under this Constitution and the Compiled Code.

    The Assembly shall produce, publish, and maintain an Assembly Register as an account of all Assembly proceedings. The Register shall minimally include minutes, attendance, reports, recorded votes, and resolutions.

    The Assembly shall produce, publish, and maintain a Complied Code of legislation, which shall be a compilation of all regulations, excluding provisions of the Operating Procedures, currently and permanently affecting student government or the student body.

     The Assembly shall have the power to pass resolutions and amend the Operating Procedures and the Compiled Code. All resolutions to amend the Compiled Code shall pass the Assembly by a simple majority vote.

     With the agreement of a simple majority of the University Council, the Assembly may place proposed amendments to this constitution before the students for ratification according to the process described in Article V of this Constitution.

    The Assembly may require a student's presence at a hearing by clear and timely subpoena.

\section{The University Council.}
    The University Council shall be composed of representatives chosen by the several degree-granting unit governments every year. One representative shall be elected by the membership of each such government from among their officers or by such other method as that government shall determine. Each representative shall have one vote in the Council. The Council shall meet at least twice monthly during the academic year.

    The Central Student Government Vice President shall serve as president of the Council, but shall have no vote, unless the Council shall be equally divided. The Council shall choose other officers as it shall deem necessary from among its members. Council officers shall be responsible for accepting items for the Council's agenda and docket, and, excepting the Vice President, officers may be recalled by a two-thirds majority vote of the Council.
 
    With the agreement of a simple majority of the Assembly, the Council may place proposed amendments to this constitution before the students for ratification according to the process described in Article V.

    Any student organization with at least 400 active members currently enrolled at the university may apply for a non-voting ex-officio seat on the Council. The Council shall create, publish, and maintain requirements for review of such applications.

    The Council shall form such committees as it deems necessary to execute its duties.

    The Council shall produce, publish, and maintain a Council Register as an account of all Council proceedings. The Register shall minimally include minutes, attendance, reports, recorded votes, and resolutions.

    The Council shall oversee the University Elections Commission for administering and certifying elections to the Central Student Government. The powers, functions, and composition of the University Elections Commission shall be defined in the Compiled Code.

    Each school, college, or organization government recognized by membership in the Council shall, from time to time, be able to recommend to the Council such measures as it deems necessary and expedient. After deliberation on such a measure, the Council may, by a simple majority, add the measure at issue to the agenda of the next Assembly meeting. No more than one such measure shall have originated from any single Council member per Council meeting.

\section{Proceedings and Membership.}
    The Assembly and University Council shall each determine rules of its own proceedings, including attendance, committee, and constituent service policies.  A simple majority of the members duly elected or appointed to serve in the Assembly shall constitute a quorum to do business.

    No elected officer of the Central Student Government shall, within his or her elected term, hold any other office in the Central Student Government.

    Any elected or appointed officer of the Central Student Government may be removed from office for delinquency, corruption, or other derelictions. Articles of impeachment must pass the Assembly by a simple majority, after which they shall be presented to the University Council for a hearing. A quorum of no less than half the number of degree-granting units of University Council members must be present to hear impeachment proceedings. When the Council is convened to hear impeachment charges, the Chief Justice of the Central Student Judiciary shall preside. At the conclusion of the hearing, a two-thirds majority of the Council shall be required for conviction. A convicted officer shall be immediately removed from office.

    If at any time a seat on the Assembly or University Council shall become vacant, the legislature of the inferior constituency controlling the seat shall appoint a new representative in such manner as it shall deem appropriate. The appointed representative shall serve until the next scheduled election, at which time the voters of that constituency shall fill the vacancy by election.

\section{Legislative Process and Veto.}
    If the Assembly passes a resolution concerning an amendment to the Complied Code or a resolution that would be otherwise binding upon students or student organizations, that resolution must then be submitted to the President. Upon receipt, the President shall sign it within one week or veto it. Vetoed resolutions shall return to the Assembly for reconsideration. After such reconsideration, the President's veto may be overridden by a  two-thirds majority of the Assembly.