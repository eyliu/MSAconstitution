\article{Initiatives and Referenda}
\section{Initiatives}
   Any action within the authority of the Assembly may be taken directly by the student body through the initiative. Initiative Petition shall state the exact legislation desired, and shall be signed by at least 1,000 current students.
   
   When the petition has been filed with the Assembly, the Assembly shall either adopt the legislation or submit it to the student body. The Assembly may in addition submit alternate legislation to the student body as a separate question. The question shall be on the adoption of the initiated legislation and a majority of those voting thereon shall be required for adoption.

   Initiated legislation adopted by the student body shall be binding on the Assembly, and the Assembly shall not legislate contrary to valid vote of the student body until the next General Election.

\section{Referendum}
   Any action taken by the Assembly may be brought before the student body for its decision in a referendum. A referendum petition shall state the exact legislation or part thereof which is to be voted upon, and shall be signed by at least 1,000 current students.

   When the petition has been filed with the Assembly, the Assembly shall either repeal the legislation cited, or submit the matter to the student body at an election. The Assembly may in addition submit an amended form of the contested legislation to the student body as a separate question.

   In the referendum, the question shall be on sustaining the action of the Assembly in adopting the legislation, and a majority of the voting thereon shall be required for adoption. This action shall be binding on the Assembly and the Assembly shall not legislate contrary to a valid vote of the student body until after the next general election.

   The referendum shall not extend to Constitutional amendments, not to the part of any appropriation that would normally have been expended by the time of the referendum, nor to elections in the Assembly authorized in this Constitution.

\section{Constitutional Conventions}
   Amendments to this Constitution may also be initiated by a vote of two-thirds present and voting of a duly called and elected Constitutional Convention. The manner of calling, electing, filling vacancies, and submitting and dividing questions, and the operating procedures for such convention, shall be specified by the Assembly. Any amendments initiated by such a convention shall be referred to the student body at an election. If three-fifths of those voting on an amendment approve it, the amendment shall be adopted.