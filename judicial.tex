\article{The Central Student Judiciary}

\section{The Central Student Judiciary}
    Judicial authority shall be vested in one Central Student Judiciary, and in such courts as degree-granting units and specific interest organizations shall create. The Central Student Judiciary shall consist of eleven Justices selected from among the students. No more than \temp{five} Justices shall be enrolled in any one degree-granting unit. The term for Justices shall be one year, and justices may serve consecutive terms. New Justices shall be recommended to the Senate for confirmation by sitting Central Student Judiciary Justices, and shall be confirmed by a simple majority vote.

    The Central Student Judiciary shall hear appeals from inferior courts, challenges to this Constitution, Central Student Government elections, matters of diverse jurisdiction, and any other case they deem appropriate. It may advise the Legislature and Executive in matters of this Constitution, the Compiled Code, or pending legislation. It shall be served by a \temp{paid Reporter}, who shall record and publish their decisions.

    \temp{The Central Student Judiciary shall write, publish, and maintain a Manual of Procedure for Appeal and Original Jurisdiction consistent with the provisions of the Constitution. The manual shall include provisions for informing a student of his or her rights, for assuring the impartiality of the panel hearing the case, and for jury trial whenever suspension or expulsion from the University is possible.}

    \temp{The Central Student Judiciary's docket, selection of Chief Justice and Student Advocates, manual, budget, and public countenance shall be overseen by the Student Assembly or any committee the Assembly shall designate for such purpose.}

    The Central Student Judiciary shall enforce no regulation inconsistent with this Constitution in content or origin.

\section{Officers and Advocates}
    Justices shall select from among their number a Chief Justice, who shall serve a one year term. No Justice shall serve more than one term as Chief Justice. The Chief Justice must have served at least one semester as a Justice before elevation to Chief Justice. The Chief Justice shall determine which Justices will hear individual cases.

    The Chief Justice shall oversee the selection of \temp{several} Student Advocates, who shall serve one year terms. Student Advocates shall meet regularly with leaders of student organizations, and hold public meetings with interested students, to advise them of students' rights and responsibilities before the Central Student Judiciary. Student Advocates shall be available to represent organizations or students before the Central Student Judiciary. 

    The Central Student Judiciary shall maintain an online record of their opinions which shall be available to all students.
