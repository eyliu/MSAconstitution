\article{The Central Student Judiciary}
\section{The Central Student Judiciary.}
    Judicial authority shall be vested in one Central Student Judiciary and in such courts as degree-granting units and specific interest organizations shall create. The Central Student Judiciary shall consist of nine Justices selected from among the students. No more than four Justices shall be enrolled in any one degree-granting unit. The term for Justices shall be one year, and justices may serve consecutive terms. New Justices shall be recommended to the Assembly for confirmation by sitting Central Student Judiciary Justices, and shall be confirmed by a simple majority vote.

    The Central Student Judiciary may hear appeals from inferior courts, challenges under this Constitution and the Compiled Code, Central Student Government elections, and any other case they deem appropriate. It shall be served by a paid Reporter, who shall maintain a public record of their opinions which shall be available to all students. The Central Student Judiciary may elect to redact any information necessary to guarantee student privacy in accordance with the Family Education Rights and Privacy Act.

   The Central Student Judiciary shall write, publish, and maintain a Manual of Procedure for Appeal and Original Jurisdiction consistent with the provisions of the Constitution. The manual shall include provisions for informing a student of his or her rights, for assuring the impartiality of the panel hearing the case, and for jury trial whenever suspension or expulsion from the University is possible.

    The Central Student Judiciary may require a student's presence at a hearing by clear and timely subpoena.

    The Central Student Judiciary shall enforce no regulation inconsistent with this Constitution in content or origin.

\section{Officers and Advocates.}
    Justices shall select from among their number a Chief Justice, who shall serve a one year term. No Justice shall serve more than one term as Chief Justice. The Chief Justice must have served at least one semester as a Justice before elevation to Chief Justice. The Chief Justice shall determine which Justices will hear individual cases. In any legislative term, the Chief Justice shall preside over the Assembly until a Speaker is elected.

    The Chief Justice shall oversee the selection of several Student Advocates, who shall serve one year terms. Student Advocates shall meet regularly with leaders of student organizations, and hold public meetings with interested students, to advise them of students' rights and responsibilities before the Central Student Judiciary. Student Advocates shall be available to represent organizations or students before the Central Student Judiciary. Student Advocates shall be justly compensated for their service. Student Advocates may advise the Legislature or any Executive Officer in matters of this Constitution, the Compiled Code, or pending legislation. Their opinions shall not bind the Central Student Judiciary or any other office of the Central Student Government.