\article{The Central Executive}
\section{The President.}
    Executive power shall be vested in a President of the Central Student Government, who shall serve as chief executive of and chief advocate for the student body. The President shall serve a one-year term and shall be elected, together with the Vice President, by a vote of the general student body. The Student Body President and Vice President shall be elected together in the March election for a one year term by a vote of the student body at large.  The method of voting for the Student Body President and Vice President shall be the method of plurality voting.  Each student will be allowed to vote for one slate, and the slate with the most votes shall be declared the winners.  In the event of a tie, the newly elected Assembly shall choose amongst the tied candidates.

    The President shall have the authority to appoint a Treasurer, Student General Counsel, Chief of Staff, and Chief Programming Officer to the Executive Committee, which shall advise the President on all pertinent matters. The President shall also appoint student representatives to university-wide committees. All Executive appointments shall be made with the advice and consent of the Assembly, to be determined by a simple majority vote. The President may likewise recall these officers with a two-thirds majority vote of the Assembly. The President may call the Executive Committee into session at any time, and shall serve as its chair in session.

    The President may appoint Executive Commissions to study issues on campus, publish reports concerning issues under such purview, and recommend to the Executive Branch such measures as they shall deem appropriate. Commissions shall serve a term of one year, but may be granted successive terms by an annual vote of the Assembly. If a Commission has been in active operation for three years, the Assembly may grant that Commission a three-year term. The President shall, with the consent of a simple majority of the Assembly, appoint Commission chairs, who shall not be considered officers of the Central Student Government. Members of the Commission may elect from among their number any other officers they deem expedient. The President may remove a Commission chair with the written concurrence of three other executives. The powers, functions, and composition of these Commissions shall be defined in the Compiled Code.

    The President may call into session the Assembly or the University Council at the President's discretion. The President shall serve as a non-voting ex-officio member of the Assembly.

    The President and Vice President may, jointly or severally, recommend to the Assembly for its consideration such measures as they shall deem appropriate. The President shall, prior to the end of each academic year, submit to the Assembly and the students at large, a report of the state of student government and of the student body.

\section{Other Executives.}
    The Vice President shall serve as a non-voting ex-officio member of the Assembly and of any Assembly committee he shall elect.

   The President shall appoint a Treasurer, who shall be the chief financial officer of the Central Student Government. The Treasurer and all other officers authorized by the Assembly to disburse funds must be bonded. The Treasurer shall disburse funds appropriated by the Assembly as provided for in this Constitution and in the Compiled Code, and shall create, publish, and maintain a manual to guide student organizations in pursuing budget allocations. The Treasurer shall, at the direction of the President, assist the legislature in drafting a proposed annual budget for the Central Student Government and present it to the Assembly for a vote. The Treasurer may serve as a non-voting ex-officio member of any legislative body regarding student finance.

    All financial records of the Assembly shall be open to public inspection. There shall be an annual audit of the finances of the Assembly, which shall be made promptly available for complete public inspection.

    The President shall appoint a Student General Counsel. The Student General Counsel shall be the chief representative of the Central Student Government in matters before student judiciaries. The Student General Counsel may retain up to three student representatives to serve as assistants in such matters. The Student General Counsel shall advise the Executive and the Legislature on the interpretation of the Constitution and the Compiled Code, and may serve as a non-voting ex-officio member of any legislative body concerning rules and elections of student government.
    
    The President shall appoint a Chief of Staff. The Chief of Staff shall oversee attendance and procedural policies at meetings of the Executive Committee and executive commission meetings. The Chief of Staff shall solicit and receive reports of the various organs of government, maintain and publish executive records, and ensure collaboration among the various executive commissions. The Chief of Staff may serve as a non-voting ex-officio member of any legislative body concerning rules and elections.

    The President shall appoint a Chief Programming Officer. The Chief Programming Officer shall serve as principal advisor to the President on matters of student programming, assist executive commissions in the long-range planning and execution of their mandate, and supervise the communications of the Central Student Government. The Chief Programming Officer may serve as a non-voting ex-officio member of any legislative body concerning campus communication.

\section{Presidential Succession.}
    If the office of Vice President, or of any Executive Office normally appointed by the President, becomes vacant, the President shall name a replacement with the advice and consent of a simple majority of the Assembly. Upon removal or incapacitation of the President, the Vice President shall assume the duties of the presidency. If both the office of President and Vice President shall be simultaneously vacant, the Speaker of the Assembly shall assume the duties of President.
